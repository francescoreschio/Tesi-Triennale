\documentclass[a4paper,11pt]{book}

\usepackage[top=2cm,bottom=2cm,left=2cm,right=2cm]{geometry}
\usepackage[pdftex]{graphicx} %per poter inserire le figure
\usepackage{amssymb,amsmath,amsthm,amsfonts}
\usepackage{xspace}
\usepackage{tabularx}
\usepackage{indentfirst}
\usepackage{subfigure}
\usepackage[small]{caption}
\usepackage{eucal}
\usepackage{eso-pic}
\usepackage{url}
\usepackage{booktabs}
\usepackage{afterpage}
\usepackage{parskip}
\usepackage{listings}
\usepackage{textcomp}
\usepackage{cite}
\usepackage{multirow}
\usepackage{subfiles} % Best loaded last in the preamble
\usepackage[utf8]{inputenc}   %per riuscire a scrivere gli accenti
\usepackage[italian]{babel}   %per riuscire a scrivere gli accenti
\usepackage{setspace} 



\begin{document}
\frontmatter
\begin{titlepage}
\vspace{5mm}
\begin{figure}[hbtp]
\centering
\includegraphics[scale=.13]{../Immagini/unipd_logo.png}
\end{figure}
\vspace{5mm}
\begin{center}
{{\huge{\textsc{\bf UNIVERSIT\`A DEGLI STUDI DI PADOVA}}}\\}
\vspace{5mm}
{\Large{\bf Dipartimento di Fisica e Astronomia ``Galileo Galilei''}} \\
\vspace{5mm}
{\Large{\textsc{\bf Corso di Laurea in Fisica}}}\\
\vspace{20mm}
{\Large{\textsc{\bf Tesi di Laurea}}}\\
\vspace{30mm}
\begin{spacing}{3}
{\LARGE \textbf{Proprietà dei candidati Muoni del Trigger L1 di CMS}}\\
\end{spacing}
\vspace{8mm}
\end{center}

\vspace{20mm}
\begin{spacing}{2}
\begin{tabular}{ l  c  c c c  cc c c c c  l }
{\Large{\bf Relatore}} &&&&&&&&&&& {\Large{\bf Laureando}}\\
{\Large{\bf Prof./Dr. Nome Cognome}} &&&&&&&&&&& {\Large{\bf Francesco La Rovere}}\\
{\Large{\bf Correlatore}}\\
{\Large{\bf Prof./Dr. Nome Cognome}}\\
\end{tabular}
\end{spacing}
\vspace{15 mm}

\begin{center}
{\Large{\bf Anno Accademico 2023/2024}}
\end{center}
\end{titlepage}
%\clearpage{\pagestyle{empty}\cleardoublepage}

\tableofcontents

\chapter*{Abstract}
Dalla primavere del 2024 a CMS e' in produzione un sistema per acquisire a 40 MHz (ovvero senza filtro di trigger) i dati relativi ad i candidati oggetti fisici ricostruiti dal sistema di trigger di primo livello; tale sistema e' indicato come L1 Trigger Data Scouting, L1DS. In particolare il L1DS raccoglie le informazioni dai vari passaggi della catena logica dedicata alla identificazione e misura dei muoni, sono in particolare a disposizioni i "segmenti" individuati da ciascuna stazione dello spettrometro e le tracce ottenute a partire da questi. Lo studio che si propone in questa tesi verte sull'analisi di questi dati, con l'obbiettivo di caratterizzarne le proprietà. Appurato che le performance siano compatibili con quanto atteso (sulla base del confronto con i dati sintetici prodotti con simulazioni Monte Carlo), si utilizzeranno questi andati per cercare tracce con uno sviluppo temporale piu' lungo dello standard (in particolare sviluppandosi su piu' "bunch crossing") al fine di mettere le basi per la ricerca di particelle esotiche "lente" ovvero prodotte con beta
non vicino ad 1.


\chapter{Introduzione}
\label{cap:Introduzione}

In questo studio sono presentate le proprietà dei candidati muoni derivanti dal sistema di Trigger di primo livello del CMS, noto come L1T. Questo ha lo scopo di filtrare gli eventi derivanti dalle collisioni di protoni in modo da ridurre il volume di dati da analizzare, mantenendo solamente gli eventi interessanti. Come sarà discusso nel Capitolo \ref{cap:PrimoCapitolo} ciò introduce un bias, ovvero un pregiudizio sui dati, mascherando possibili informazioni che potrebbero portare alla scoperta di fisica Oltre il Modello Standard. A questo scopo viene introdotto il sistema di Data Scouting nel L1T, ovvero un sistema che consente di raccogliere eventi derivanti dalle collisioni, seppur con una minore risoluzione, eseguendo una analisi a livello del L1T e lavorando parallelamente ad esso. Studiare questi eventi piuttosto che quelli in uscita dal Trigger comporta ovviamente una maggiore presenza di segnali di fondo, ma ciò viene eseguito senza introdurre nessun bias nei dati analizzati per la analisi. Il sistema di Data Scouting nel L1T verrà implementato definitivamente a CMS con l'upgrade di LHC, Phase 2, permettendo quindi un grosso passo in avanti verso la scoperta di Nuova Fisica. In previsione della Phase 2, durante la Run 3 a CMS è stato introdotto un sistema di Data Scouting apposito a livello del L1T che consente la validazione e la sperimentazione di nuovi algoritmi da implementare con l'upgrade di CMS.


Nel Capitolo \ref{cap:PrimoCapitolo} verrà quindi introdotto il Large Hadron Collider e il suo principale esperimento, il Compact Muon Solenoid locato a Cessy, in Francia nel punto di interazione 5. Particolare attenzione verrà posta sulle camere muoniche, che permettono la rilevazione di muoni a CMS, e sul sistema di Trigger. Verrà inoltre più dettagliatamente discusso il sistema di Data Scouting di CMS nella Sezione \ref{sec:DataScouting}. Particolare attenzione avrà anche la sezione sulla ricerca di Nuova Fisica \ref{sec:NewPhysics}.

Nel Capitolo \ref{cap:SecondoCapitolo} verranno presentati e validati i risultati del sistema di Data Scouting introdotto con la Run 3, studiando le informazioni rilevate dalle schede di acquisizione nei principali step di acquisizione, verificandone la conformità. Verrà inoltre eseguito uno studio approfondito sul confronto tra candidati muoni del sistema di tracking BMTF e i muoni del GMT, esaminando quindi le differenze. Alla fine del Capitolo \ref{cap:SecondoCapitolo} verrà presentato un breve confronto con i dati raccolti dal sistema di Data Scouting nell'anno 2023 \cite{CERNsummerSchool}, evidenziandone le principali differenze.

Nel Capitolo \ref{cap:TerzoCapitolo} verrà invece introdotto l'algoritmo utilizzato per la ricerca di eventi compatibili con i modelli alla base delle Heavy Stable Charged Particles, mostrando anche i principali risultati ottenuti.

Infine nel Capitolo \ref{cap:Conclusioni} verranno presentate le conclusioni ottenute dai risultati del presente studio. 



\chapter{Primo Capitolo:}
\section{1}

\appendix



\end{document}






