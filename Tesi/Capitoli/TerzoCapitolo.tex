\chapter{Ricerca Heavy Stable Charged Particles}
\label{cap:TerzoCapitolo}

\section{Ricerca di Nuova Fisica al CMS}
\label{sec:NewPhysics}

Proposto inizialmente nel 1961 da Sheldon Glashow, e raffinato da Steven Weinberg e Abdus Salam nel 1968, il Modello Standard (SM) descrive le tre interazioni (Forte, Debole ed Elettromagnetica) che agiscono tra le particelle fondamentali che costituiscono la materia. Il beneficio di avere un frame completo come il Modello Standard risiede nella capacità di prevedere il comportamento di particelle subatomiche conoscendo la struttura teorica alla base. Una delle maggiori conquiste dello SM è la scoperta del bosone di Higgs, teorizzato per la prima volta da Higgs nel 1964 e rilevato a CMS nel 2012. \newline
Nonostante negli ultimi 50 anni molte siano le conferme sperimentali del Modello Standard, ci sono fenomeni che non possono essere spiegati esaustivamente dallo stesso e questo suggerisce la presenza di fisica oltre il Modello Standard (BSM).

Diversi modelli di fisica oltre il Modello Standard suggeriscono la presenza di particelle cariche \textit{longeve} con masse di svariate centinaia di \si{GeV/c^2}, chiamate Heavy Stable Charged Particles (HSCPs). I modelli prevedono la presenza di due principali categorie di HSCPs: di tipo \textit{leptonico} o di tipo \textit{adronico}. Generalmente questi ultimi sono chiamati adroni-R (R-hadrons) \cite{Quertenmont:2010ota}. \newline
Come gli adroni, gli adroni-R possono subire fenomeni di scattering da parte dei nuclei del materiale di cui sono formati i detector (scattering adronici). I modelli teorici suggeriscono le principali firme sperimentali delle HSCPs sono una anormale perdita di energia per unità di lunghezza $- \langle \frac{dE}{dx}\rangle$ e un maggiore tempo di volo (ToF) rispetto a particelle standard \cite{MasterThesisGioMoc}, riconducibili ad una velocità minore rispetto alla velocità della luce ($\beta < 1$). Inoltre le HSCPs sono particelle estremamente penetranti, ciò indica che queste si comportino come muoni e per tanto possano essere rilevate nelle camere muoniche. \newline
Essendo inoltre particelle cariche con un elevato tempo di vita medio (maggiore di 1ns) sono in grado di attraversare i rilevatori prima di decadere, producendo una traccia di ionizzazione nei detector.


Prima della Phase 1 al CMS era implementato un sistema di trigger specifico per la ricerca di particelle massive con un lungo tempo di volo e una velocità molto minore della velocità della luce. Ciò era possibile in quanto, con una minore luminosità, vi era mediamente una collisione ogni 50ns. Con la Phase 1 e quindi con un aumento della luminosità istantanea, ovvero un minore tempo di Bunch Crossing (uno ogni 25ns) il sistema di trigger per particelle esotiche è stato rimosso \cite{MasterThesisGioMoc}