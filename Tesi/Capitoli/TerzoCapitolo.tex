\chapter{Nuova Fisica a CMS}
\label{cap:TerzoCapitolo}

\section{Ricerca di HEavy Stable Charged Particles}
\label{sec:NewPhysics}

Proposto inizialmente nel 1961 da Sheldon Glashow, e raffinato da Steven Weinberg e Abdus Salam nel 1968, il Modello Standard (SM) descrive le tre interazioni (Forte, Debole ed Elettromagnetica) che agiscono tra le particelle fondamentali che costituiscono la materia. Il beneficio di avere un frame completo come il Modello Standard risiede nella capacità di prevedere il comportamento di particelle subatomiche conoscendo la struttura teorica alla base. Una delle maggiori conquiste dello SM è la scoperta del bosone di Higgs, teorizzato per la prima volta da Higgs nel 1964 e rilevato a CMS nel 2012. \newline
Nonostante negli ultimi 50 anni molte siano le conferme sperimentali del Modello Standard, ci sono fenomeni che non possono essere spiegati esaustivamente dallo stesso e questo suggerisce la presenza di fisica oltre il Modello Standard (BSM).

Diversi modelli di Fisica oltre il Modello Standard suggeriscono la presenza di particelle cariche longeve (\textit{Long-Lived}) con masse di svariate centinaia di \si{GeV/c^2}, chiamate Heavy Stable Charged Particles (HSCPs). Se il tempo di decadimento di queste particelle è abbastanza elevato, esse potrebbero attraversare completamente il detector prima ancora di decadere: la ricerca di particelle longeve, e in particolare di HSCP, nei moderni rilevatori di particelle richiedono strategie diverse rispetto a quelle impiegate per la ricerca di altre particelle appartenenti a modelli BSM. \newline 
Un'importante firma sperimentale di HSCP è l'elevata perdita di energia per unità di lunghezza $\langle \frac{dE}{dx}\rangle$ nei materiali del detector, dipendente dalla velocità e dalla carica delle particelle incidenti, in particolare dalla relazione di Bethe-Bloch, minore è la velocità delle particelle che attraversano un materiale, maggiore è l'energia persa per unità di lunghezza sotto forma di ionizzazione: ciò porta a supporre che le HSCP siano particelle con una velocità molto minore rispetto alla velocità della luce ($\beta < 1$). 



È stato discusso in Sezione \ref{sec:SistemaDiTrigger} e nel Capitolo \ref{cap:SecondoCapitolo} come il sistema di Trigger di CMS sia necessario per ridurre il volume di informazioni derivanti dalle collisioni di eventi e di come il L1T applichi un primo filtro per la selezione di eventi interessanti tra i numerosi eventi di background. È stato anche trattato come il sistema di Trigger possa celare eventi compatibili con i diversi modelli di Nuova Fisica: nonostante nel sistema di Trigger permetta la ricerca di fenomeni interessanti Oltre il Modello Standard, questi processi potrebbero essere talmente rari che sarebbero necessari sistemi di ricerca specializzati.\newline
Prima della Phase 1 di CMS era implementato un sistema di trigger specifico per la ricerca di particelle massive con un lungo tempo di volo e una velocità molto minore della velocità della luce. Ciò era possibile in quanto, con una minore luminosità, vi era mediamente una collisione ogni 50ns. Con la Phase 1 e quindi con un aumento della luminosità istantanea, ovvero un minore tempo di Bunch Crossing (uno ogni 25ns) il sistema di trigger per particelle esotiche è stato rimosso \cite{MasterThesisGioMoc}.\newline
Come discusso in \ref{sec:Phase2} il CMS subirà però con la Phase 2 importanti miglioramenti al sistema di Trigger e di tracciamento che permetteranno di ricavare informazioni unbiased direttamente dalla catena di Trigger a con massima risoluzione. In particolare quindi il sistema di Data Scouting al L1T fornirà sarà un grosso passo avanti nella ricerca di Nuova Fisica, nel contesto nella ricerca di HSCP. \newline







%Nel contesto della ricerca di Heavy Stable Charged Particles a CMS sfrutta questi due importanti proprietà delle particelle: una velocità molto ridotta e, conseguente al maggior tempo di vita medio, un maggiore tempo di volo (ToF) rispetto a particelle standard. Pertanto vi sono due strategie per la ricerca di HSCP: la prima sfrutta le informazioni e l'analisi dei segnali del sistema di tracciamento, concentrandosi sulla elevata perdita di energia delle particelle (tracker), mentre la seconda strategia si basa sulla analisi combinata nelle camere muoniche dei segnali ad elevata ionizzazione e sull'elevato tempo di volo (tracker + ToF) \cite{Lenz:2214515}

%I modelli prevedono la presenza di due principali categorie di HSCPs: di tipo \textit{leptonico} o di tipo \textit{adronico}. Generalmente questi ultimi sono chiamati adroni-R (R-hadrons) \cite{Quertenmont:2010ota}. \newline
%Come gli adroni, gli adroni-R possono subire fenomeni di scattering da parte dei nuclei del materiale di cui sono formati i detector (scattering adronici). I modelli teorici suggeriscono le principali firme sperimentali delle HSCPs sono una anormale perdita di energia per unità di lunghezza $- \langle \frac{dE}{dx}\rangle$ e un maggiore tempo di volo (ToF) rispetto a particelle standard \cite{MasterThesisGioMoc}, riconducibili ad una velocità minore rispetto alla velocità della luce ($\beta < 1$). Inoltre le HSCPs sono particelle estremamente penetranti, ciò indica che queste si comportino come muoni e per tanto possano essere rilevate nelle camere muoniche. \newline
%Essendo inoltre particelle cariche con un elevato tempo di vita medio (maggiore di 1ns) sono in grado di attraversare i rilevatori prima di decadere, producendo una traccia di ionizzazione nei detector.



