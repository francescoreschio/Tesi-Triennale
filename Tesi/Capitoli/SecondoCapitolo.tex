\chapter{Verifica della tecnica di Data Scouting:}
\label{cap:SecondoCapitolo}

Nella Sezione \ref{sec:SistemaDiTrigger} si è discusso nel dettaglio del sistema di Trigger Level 1 di CMS e di come questo giochi un ruolo fondamentale nella selezione di eventi interessanti a seguito della collisione tra protoni. In particolare è stato osservato che le informazioni provenienti dai detector locali nella regione di barrel del sistema muonico vengono combinate e processate dalle schede TwinMux, generando \textit{superprimitives}, o stubs, con una risoluzione temporale e spaziale molto elevata. Le superprimitives vengono poi utilizzate dal sistema di tracking (BMTF) per ricostruire la traccia del muone.

Inoltre nella Sezione \ref{sec:DataScouting} si è introdotto il sistema di Data Scouting impiegato durante la Run 3 nel L1T, che permette di analizzare, seppur con una minore risoluzione, eventi unbiased che verrebbero potenzialmente rigettati dal sistema di Trigger. Per dimostrare la validità del sistema di Scouting nel L1T viene implementato un sistema di acquisizione dati a vari step della catena, in particolare a livello del Barrel Muon Track Finder, del Global Muon Trigger e del Global Trigger.

In questo capitolo verranno studiati i dati ottenuti dal sistema di Data Scouting, validandone la conformità.

\section{Validazione superprimitives}

Tra la Run 1 e la Run 2, sono state introdotte nel sistema di Trigger L1 le schede TwinMux nella catena di Trigger. Queste hanno il compito di generare stubs a partire dai segnali in input derivanti da Drift Tubes e Resistive Plate Chambers. \newline
In particolare vengono applicati algoritmi di clustering ai dati in input alle schede, convertendo i segnali spaziali delle RPC nelle coordinate dei DT. Se i segnali provenienti dai due rilevatori sono compatibili, ovvero se $\Delta \phi < 15$ \si{mrad}, allora vengono combinati. Nel dettaglio i rilevatori DT forniscono informazioni sulla posizione, direzione, qualità e Bunch Crossing (BX) ad un rate di 480Mb/s, mentre le RPC sulla posizione e sul BX con un rate di 1.6Gb/s.\newline
È funzione del TwinMux assegnare dei \textit{parametri} a ciascuna stub generata che permettono al BMTF di ricostruire la traccia del muone a partire da queste informazioni. I parametri che vengono assegnati ad ogni stub riguardano la posizione spaziale nel CMS, l'angolo di curvatura relativo alla stazione e la qualità della stub. Questi vengono riassunti nella Tabella \ref{table:StubsParam}.

Il sistema TwinMux può costruire fino a 2 superprimitives per stazione; queste vengono poi inviate al BMTF e, in base alla posizione nel piano della pseudorapidità, all' OMTF. Di seguito l'analisi di Data Scouting degli input del sistema di tracking verrà effettuato solamente per eventi in ingresso al Barrel Muon Track Finder.

Questa sezione è il primo passo alla validazione del sistema di Data Scouting introdotto con la Run 3 nel L1T: vengono studiate le stubs, ovvero i segnali di ingresso al BMTF. In particolare verranno utilizzati 1 minuto e 32 secondi di presa dati, corrispondenti a circa $1.1 \times 10^{6}$ stubs. 

Come illustrato nella Sezione \ref{sec:LHC} in LHC circolano fasci di protoni che collidono nei punti di interazione dove sono presenti i principali esperimenti. Per ottimizzare e massimizzare le collisioni nei punti di interazione i pacchetti di protoni vengono disposti in modo specifico, facendo in modo che ogni fascio può contenere al massimo 2808 pacchetti di protoni, nonostante ci siano 3564 posizioni disponibili lungo l'anello di LHC. La disposizione dei pacchetti di protoni nei fasci è chiamata \textit{filling scheme} e generalmente questa può variare in base all'esperimento. Durante la Run 3 il filling scheme utilizzato è mostrato in Figura \ref{fig:StubsBX}, formato da 39 gruppi da 72 pacchetti di protoni \cite{Bailey}. \newline
Si nota come il rate di stubs non è mai nullo e vi sia sempre un fondo costante di eventi. Questo fenomeno è da attribuire alla presenza di eventi di background come muoni cosmici, che vengono rilevati dai detector di CMS e vengono trattati, almeno in prima analisi, come muoni generati dalla collisione di protoni.


 \begin{figure}[t]
   \centering
   \begin{minipage}[b]{0.49\textwidth}
       \centering
       \includegraphics[width=\textwidth]{../Immagini/StubsBXnumber.pdf} 
     \end{minipage}
     \hfill 
     \begin{minipage}[b]{0.49\textwidth}
       \centering
       \includegraphics[width=\textwidth]{../Immagini/StubsBXnumberZoom.pdf} 
   \end{minipage}
   \caption{Filling scheme per la Run 3: disposizione dei pacchetti di protoni nelle 3564 possibili posizioni }
   \label{fig:StubsBX}
 \end{figure}









 \renewcommand{\arraystretch}{1.25} % Aumenta lo spazio tra le righe
 \begin{table}[h!]
   \centering
   \begin{tabular}{cccl}
   \toprule
   \textbf{Parameter} & \textbf{Bits} & \textbf{Range} & \textbf{Description} \\ \hline
   $\phi$            & 12  & [$-2048$, 2047]   & Relative position of a segment inside a sector \\ \hline
   $\phi_b$          & 10  & [$-512$, 511]     & Bending angle \\ \hline
   \textit{quality}  & 3   & [0, 7]            & Number of superlayers used to construct the stub \\ \hline
   $\eta$ hits       & 7   & "pattern"         & \begin{tabular}[c]{@{}l@{}} Each bit corresponds to one chamber area \\ 0 : no hit (less than 3 SL hits) \\ 1 : hit (3 or 4 SL hits) \end{tabular} \\ \hline
   $\eta$ quality    & 7   & "pattern"         & \begin{tabular}[c]{@{}l@{}} Each bit corresponds to one chamber area \\ 0 : 3 SL hits \\ 1 : 4 SL hits \end{tabular} \\ 
   \bottomrule
   \end{tabular}
   \caption{Descrizione dei parametri assegnati dal TwinMux a ciascuna stub}
   \label{table:StubsParam}
 \end{table}


\section{Validazione dell'emulatore Barrel Muon Track Finder}

\section{Validazione Global Muon Trigger}

