\chapter{Proprietà dei candidati muoni:}
\label{cap:SecondoCapitolo}

Nella Sezione \ref{sec:SistemaDiTrigger} si è discusso nel dettaglio del sistema di Trigger Level 1 di CMS e di come questo giochi un ruolo fondamentale nella selezione di eventi interessanti a seguito della collisione tra protoni, riducendo il volume di dati da analizzare. In particolare le informazioni provenienti dai detector locali del sistema muonico nella regione di barrel vengono combinate e processate dalle schede \textbf{TwinMux}, generando \textit{superprimitives}, o stubs, con una risoluzione temporale e spaziale elevata. Le superprimitives vengono poi utilizzate dal sistema di tracking nella zona di barrel (BMTF) per ricostruire la traccia del candidato muone.

Inoltre nella Sezione \ref{sec:DataScouting} si è introdotto il sistema di Data Scouting impiegato durante la Run 3 nel L1T che permette di analizzare, seppur con una minore risoluzione, eventi unbiased che verrebbero potenzialmente rigettati dal sistema di Trigger. In questo capitolo si procederà con lo studio e la validazione dei dati ottenuti dal sistema di Data Scouting, che raccoglie informazioni impiegando schede di acquisizione in vari step della catena di Trigger, in particolare a livello degli input del Barrel Muon Track Finder, del Global Muon Trigger e del Global Trigger.


\section{Validazione superprimitives}
\label{sec:Superprimitives}

Tra la Run 1 e la Run 2, sono state introdotte nel sistema di Trigger L1 le schede TwinMux nella catena di Trigger: queste hanno il compito di generare superprimitives (da questo momento in poi \textit{stubs}) a partire dai segnali in input derivanti da Drift Tubes e Resistive Plate Chambers. \newline
In particolare sono applicati algoritmi di clustering ai dati in input alle schede, convertendo i segnali spaziali delle RPC nelle coordinate dei DT. Se i segnali provenienti dai due rilevatori sono compatibili, ovvero se $\Delta \phi < 15$ \si{mrad}, allora vengono combinati. Nel dettaglio i rilevatori DT forniscono informazioni sulla posizione, direzione, qualità e Bunch Crossing (BX) ad un rate di 480Mb/s, mentre le RPC sulla posizione e sul BX con un rate di 1.6Gb/s \cite{CERNsummerSchool}.\newline
Il sistema TwinMux può costruire fino a 2 stubs per stazione; queste vengono poi inviate al BMTF e, in base alla posizione nel piano della pseudorapidità, all' OMTF. \newline
È inoltre funzione delle schede TwinMux assegnare dei \textit{parametri} a ciascuna stub generata: questi riguardano la posizione spaziale nel CMS, l'angolo di curvatura relativo alla stazione e la qualità della stub e sono riassunti nella Tabella \ref{table:StubsParam}. A partire da queste informazioni il BMTF ricostruisce la traccia del muone, determinando la traiettoria della particella. 

Questa sezione è il primo passo alla validazione del sistema di Data Scouting introdotto con la Run 3 nel L1T: vengono studiate le stubs, ovvero i segnali di ingresso al BMTF. In particolare verranno utilizzati 1 minuto e 32 secondi di presa dati, corrispondenti a circa $1.1 \times 10^{6}$ stubs. 

Come illustrato nella Sezione \ref{sec:LHC} in LHC circolano fasci di protoni che collidono nei punti di interazione dove sono presenti i principali esperimenti. Per ottimizzare e massimizzare le collisioni nei punti di interazione i pacchetti di protoni vengono disposti in modo specifico, facendo in modo che ogni fascio possa contenere al massimo 2808 pacchetti di protoni, nonostante ci siano 3564 posizioni disponibili lungo l'anello di LHC. La disposizione dei pacchetti di protoni nei fasci è chiamata \textit{filling scheme} e generalmente questa può variare in base all'esperimento. Il filling scheme utilizzato durante la presa dati è mostrato in Figura \ref{fig:Stubs1}, formato da 39 gruppi da 72 pacchetti di protoni a distanza 25ns l'uno dall'altro\cite{Bailey}. \newline
Si nota come il rate di stubs non è mai nullo e vi sia sempre un fondo costante di eventi. Questo fenomeno è da attribuire alla presenza di eventi di background come muoni cosmici, che vengono rilevati dai detector di CMS e vengono trattati, almeno in prima analisi, come muoni generati dalla collisione di protoni.



 \begin{figure}[t]
   \centering
   \begin{minipage}[b]{0.49\textwidth}
       \centering
       \includegraphics[width=\textwidth]{../Immagini/StubsBXnumber.pdf} 
     \end{minipage}
     \hfill 
     \begin{minipage}[b]{0.49\textwidth}
       \centering
       \includegraphics[width=\textwidth]{../Immagini/StubsBXnumberZoom.pdf} 
   \end{minipage}
   \caption{Filling scheme per la Run 3: disposizione dei pacchetti di protoni nelle 3564 possibili posizioni }
   \label{fig:Stubs1}
 \end{figure}


 Viene studiata la distribuzione di stubs per Bunch Crossing (BX) e per Orbita (orbit). Ricordando che il BX è l'unità di misura che definisce l'intervallo temporale tra le collisioni, in Figura \ref{fig:StubsMolteplicity} viene mostrato il numero di stubs raccolte in un BX, ovvero in 25ns, mentre in Figura \ref{fig:StubsPerOrbit} è rappresentato il numero di stubs raccolte in ogni orbita, ovvero un ciclo completo di LHC. Ogni orbita è formata da 3564 BX, ovvero circa 89 $\mu$s.

 \begin{figure}[b]
  \centering
  \begin{minipage}[b]{0.49\textwidth}
      \centering
      \includegraphics[width=\textwidth]{../Immagini/StubsMolteplicity.pdf} 
      \caption{Stubs multiplicity per BX}
      \label{fig:StubsMolteplicity}
    \end{minipage}
    \hfill 
    \begin{minipage}[b]{0.49\textwidth}
      \centering
      \includegraphics[width=\textwidth]{../Immagini/StubsPerOrbit.pdf} 
      \caption{Stubs multiplicity per orbit}
      \label{fig:StubsPerOrbit}
  \end{minipage}
\end{figure}


 È importante sottolineare che la maggior parte dei BX non contiene nessuna stubs, anche se questa informazione non è mostrata nell'immagine. Inoltre vi sono eventi che presentano un numero di stub raccolte per BX maggiore di 13, ma la frequenza di questi eventi è minore di 0.1 al secondo e per tanto non sono mostrati. \newline
 Rappresentato su scala logaritmica, la distribuzione di stubs per orbita segue una distribuzione gaussiana centrata su 230 stubs. 



  Infine viene rappresentata la distribuzione spaziale di stubs nel corpo di CMS in Figura \ref{fig:StubsInCMS} per verificare se vi sono anormalità nel sistema di acquisizione dati. Come specificato in Sezione \ref{sec:CMSDescrizione} e Figura \ref{fig:SectorEtaView}, CMS è formato da cinque ruote (\textit{wheel}), ognuna suddivisa in dodici settori (\textit{sector}) a loro volta contenenti quattro stazioni (\textit{station}). Convenzionalmente le wheel sono nominate da -2 a 2, dove 0 rappresenta la ruota in corrispondenza al punto di collisione dei protoni (origine degli assi), i sector sono nominati da 0 a 11 e le stazioni da 1 a 4. La Figura \ref{fig:StubsInCMS} mostra quindi la distribuzione spaziale delle stubs: nel grafico di sinistra notiamo una più alta densità di stubs nelle wheel $\pm$ 1, mentre una minore porzione di eventi vengono raccolti nelle wheel $\pm$ 2. Questo è dovuto al fatto che in corrispondenza di queste wheel ci troviamo nella regione ad alta pseudorapidità e per tanto vicini alla regione di \textit{overlap}: gli eventi in questa regione è probabile che vengano raccolti dal OMTF e dunque parzialmente schermati al BMTF. Nella figura di destra invece notiamo una maggiore densità in corrispondenza della prima stazione per le wheel $\pm$ 1, mentre vi è una completa assenza di eventi nella stessa stazione per le wheel $\pm$ 2. Questo difetto è attualmente noto e si verifica poiché stubs in questa stazione vengono raccolte solamente dal sistema di tracking dell'overlap, non venendo lette dal BMTF.




  \begin{figure}[t]
    \centering
    \begin{minipage}[b]{0.49\textwidth}
        \centering
        \includegraphics[width=\textwidth]{../Immagini/StubsSectorWheel.pdf} 
      \end{minipage}
      \hfill 
      \begin{minipage}[b]{0.49\textwidth}
        \centering
        \includegraphics[width=\textwidth]{../Immagini/StubsStationWheel.pdf} 
      \end{minipage}
      \caption{Distribuzione stubs nel corpo di CMS come grafico bidimensionale}
    \label{fig:StubsInCMS}
  \end{figure}





 \renewcommand{\arraystretch}{1.25} % Aumenta lo spazio tra le righe
 \begin{table}[b]
  \caption{Descrizione dei parametri assegnati dal TwinMux a ciascuna stub}
  \label{table:StubsParam}
   \centering
   \begin{tabular}{cccl}
   \toprule
   \textbf{Parameter} & \textbf{Bits} & \textbf{Range} & \textbf{Description} \\ \hline
   $\phi$            & 12  & [$-2048$, 2047]   & Relative position of a segment inside a sector \\ \hline
   $\phi_b$          & 10  & [$-512$, 511]     & Bending angle \\ \hline
   \textit{quality}  & 3   & [0, 7]            & Number of superlayers used to construct the stub \\ \hline
   $\eta$ hits       & 7   & "pattern"         & \begin{tabular}[c]{@{}l@{}} Each bit corresponds to one chamber area \\ 0 : no hit (less than 3 SL hits) \\ 1 : hit (3 or 4 SL hits) \end{tabular} \\ \hline
   $\eta$ quality    & 7   & "pattern"         & \begin{tabular}[c]{@{}l@{}} Each bit corresponds to one chamber area \\ 0 : 3 SL hits \\ 1 : 4 SL hits \end{tabular} \\ 
   \bottomrule
   \end{tabular}
   
 \end{table}


\section{Validazione candidati muoni del BMTF}
\label{sec:BMTF}

Il passo successivo nella catena di identificazione dei muoni viene eseguito dal sistema di tracking che ricostruisce la traccia del \textit{candidato} muone a partire dalle informazioni inoltrate dalle schede TwinMux. In questo studio verranno considerate solo le tracce di muoni nella zona di barrel, pertanto l'analisi verrà eseguita sulle informazioni nella regione di pseudorapidità $|\eta| < 1.2$: come già specificato nella Sezione \ref{sec:SistemaDiTrigger} il Barrel Muon Track Finder è il sistema che si occupa della ricostruzione delle tracce in questa regione.

Il BMTF implementa specifici algoritmi via hardware che permettono di convertire efficacemente le stubs generate delle schede TwinMux in candidati muoni. Dalla Tabella \ref{table:StubsParam}, ogni stub possiede informazioni spaziali e di qualità assegnate dal sistema TwinMux: in particolare 12 bit sono riservati alle coordinate $\phi$, 10 bit all'angolo di inclinazione $\phi_b$ e 3 bit alla qualità della stub. Inoltre sono riservati 7 bit per i segnali rilevati per regione di $\eta$ e 7 per la qualità di $\eta$. Di seguito verrà discusso del sistema di tracking che utilizzano le schede BMTF per ricostruire la traccia del candidato muone e del sistema di \textit{emulazione} di CMS via software che permette, attraverso specifici parametri, di emulare i risultati ottenuti via hardware e viceversa.
%In questo capitolo si valideranno le proprietà dei candidati muoni emulati con il software, 


\subsection{Ricostruzione hardware dei muoni}
\label{sec:KalmannFilter}

La ricostruzione delle tracce di muoni via hardware a CMS si sviluppa principalmente in tre step: propagazione dei parametri delle tracce ricevute dal sistema di TwinMux, ricerca di segnali vicini compatibili con la traccia e aggiornamento dei parametri di ricostruzione utilizzando \textit{Kalman filter} \cite{Summers:2728522}. \newline
Questo algoritmo, inventato da R.E. Kalman nel 1960, consente di descrivere ricorsivamente lo stato di un sistema dinamico, minimizzando il rumore e l'incertezza delle misurazioni: questo lo rende un'ottima scelta negli algoritmi di ricostruzione, in particolare a CMS \cite{welch1995introduction}. L'algoritmo di ricostruzione che usa Kalman filter è in funzione in CMS dal 2018 ed è stato rinominato kBMTF. 

Nel dettaglio, il vettore di stato $x_n = (k, \phi, \phi_b)$ rappresenta i parametri della traccia ad ogni stazione; definiamo $k = q/p_T$, dove $q$ è la carica del muone in esame e $p_T$ il momento trasverso. Il sistema di tracking utilizzato con il Kalman filter considera inizialmente le stubs nelle stazioni più esterne, propagando la traccia verso l'interno del CMS usando la seguente equazione:

\begin{equation}
  \label{eq:KalmanFilterEq}
    x_n = F x_{n-1}
\end{equation}

Dove $F$ è una matrice che descrive la geometria e l'intensità del campo magnetico del detector in ogni stazione.
Dopo di che la stub più vicina viene identificata, $z_n = (\phi, \phi_b)$ e quindi il vettore di stato viene aggiornato, $\bar{x}_n$ e viene calcolato il residuo $r_n$ tra il vettore di stato $x_n$ e la stub $z_n$: 

\begin{equation}
  \label{eq:KalmanFilterEq2}
    r_n = z_n - Hx_n = \left( \begin{matrix}
      \phi \\
      \phi_b
    \end{matrix}\right)_n - \left( \begin{matrix}
      0 & 1 & 0 \\ 
      0 & 0 & 1
    \end{matrix}\right)
    \left(\begin{matrix}
      k \\
      \phi \\
      \phi_b
    \end{matrix}\right)_n
\end{equation}

\begin{equation}
  \label{eq:KalmanFilterEq3}
    \bar{x}_n = x_n + Gr_n
\end{equation}

Dove $G$ è la matrice di Gain (Kalman Gain Matrix) e dipende dalle incertezze della previsione della misurazione: se la misurazione ha basso rumore allora il guadagno sarà alta, mentre se il rumore è alto il guadagno sarà basso. Senza entrare nel dettaglio della costruzione della matrice di Gain, si intuisce che questa dipende dalla matrice di covarianza alla iterazione $n$. Dando solo la definizione, la matrice di covarianza è la seguente:

\begin{equation}
  \label{eq:covariance}
  P_n = FP_{n-1}F^T + Q(k, x/X_0)
\end{equation}

Se nello stesso BX ci sono più tracce di muoni il Kalman filter considera solamente la traccia con $\chi^2$ minore, minimizzando quindi l'errore quadratico medio \cite{Summers:2728522}

\subsection{Ricostruzione software dei muoni}

Il Kalman filter è quindi un algoritmo che viene applicato per la ricostruzione di muoni nella catena di Trigger e per tanto deve soddisfare i requisiti di latenza richiesti dal Trigger, oltre che i limiti di banda di archiviazione. Questo non permette il passaggio di dati a risoluzione massima di alcune quantità forzando il sistema a troncare, o approssimare, le informazioni. Il sistema di emulazione del kBMTF permette quindi una analisi più fine rispetto alla analisi online, evitando quindi le limitazioni introdotte dal sistema di Trigger. Il software utilizzato per l'emulazione del Kalman filter è chiamato CMS Software (CMSSW) e permette ri ricavare informazioni circa il momento trasverso $p_T$, angolo azimutale $\phi$ e pseudorapidità $\eta$, oltre che altri dati.

In questa sezione verranno studiati i grafici di occupazione e di molteplicità dei candidati muoni emulati dal software di CMS, verificando quindi le differenze rispetto ai grafici della Sezione \ref{sec:Superprimitives}.



\begin{figure}[t]
  \centering
  \begin{minipage}[b]{0.49\textwidth}
      \centering
      \includegraphics[width=\textwidth]{../Immagini/BMTF.pdf} 
    \end{minipage}
    \hfill 
    \begin{minipage}[b]{0.49\textwidth}
      \centering
      \includegraphics[width=\textwidth]{../Immagini/BMTF_zoom.pdf} 
    \end{minipage}
    \caption{Filling Scheme dei candidati muoni emulati dal CMSSW}
  \label{fig:BMTFMuons}
\end{figure}

\begin{figure}[b]
  \centering
  \begin{minipage}[b]{0.49\textwidth}
      \centering
      \includegraphics[width=\textwidth]{../Immagini/BMTF_Molteplicity.pdf} 
    \end{minipage}
    \hfill 
    \begin{minipage}[b]{0.49\textwidth}
      \centering
      \includegraphics[width=\textwidth]{../Immagini/BMTF_orbit.pdf} 
    \end{minipage}
    \caption{Molteplicità dei candidati muoni emulati dal CMSSW per BX (sinistra) e per orbita (destra)}
  \label{fig:BMTFMolteplicity}
\end{figure}

In Figura \ref{fig:BMTFMuons} viene mostrato il filling scheme nei 3564 Bunch Crossing. Notiamo un minor rate di eventi rispetto a quello mostrato in Figura \ref{fig:Stubs1} in quanto questo filling scheme rappresenta i \textit{candidati muoni}, costruiti a partire dalle stubs delle schede TwinMux e non tutte le stubs sono associate a muoni. Ciò è reso visibile nella Figura \ref{fig:BMTFMolteplicity} sulla sinistra: la maggior parte dei BX presenta un numero nullo di muoni. Inoltre, il numero di BX contenenti un numero non nullo di muoni sembra decrescere esponenzialmente all’aumentare della molteplicità.
%Particolare attenzione deve essere posta anche sul numero di eventi di fondo: rispetto a quanto osservato nella sezione precedente, il rate di questi eventi è notevolmente inferiore. 

Sempre in Figura \ref{fig:BMTFMolteplicity}, sulla destra è mostrata la distribuzione in scala logaritmica dei candidati muoni per orbita: notiamo che la distribuzione è sempre gaussiana, centrata a circa 30 muoni ma, rispetto a quanto mostrato nella sezione precedente, vi è una asimmetria della curva che la rende inclinata positivamente, mentre la distribuzione in Figura \ref{fig:StubsMolteplicity} è simmetrica.

\begin{figure}[t]
  \centering
  \begin{minipage}[b]{0.51\textwidth}
    \centering
    \includegraphics[width=\textwidth]{../Immagini/BMTFPhiEta.pdf} 
    \end{minipage}
    \hfill 
    \begin{minipage}[b]{0.48\textwidth}
      \centering
      \includegraphics[width=\textwidth]{../Immagini/BMTF_PtPtu.pdf} 
    \end{minipage}
    \caption{Distribuzione dei candidati muoni come grafico bidimensionale in $\eta$ e $\phi$ (sinistra), distribuzione del momento trasverso unconstrained e vertex-constrained (destra)}
  \label{fig:BMTFPt}
\end{figure}

Il candidato muone ricostruito a partire dall'algoritmo kBMTF, come anche quello emulato dal CMSSW, possiede informazioni sulla posizione spaziale nelle coordinate di CMS: viene quindi assegnato un valore $\eta$ ed un valore $\phi$ ad ogni candidato muone. In figura \ref{fig:BMTFPt}, sulla sinistra, è rappresentata la distribuzione bidimensionale di eventi nelle due coordinate per verificare la conformità dei rilevatori in $\phi$ ed $\eta$. È evidente che lungo il piano $\eta = k$, con $k$ una costante, non vi siano particolarità: come ci si aspetta essendo il sistema a simmetria cilindrica la distribuzione degli eventi è costante. Si nota però una notevole asimmetria lungo $\phi = k$: ciò è dovuto alla geometria del rilevatore CMS. Come evidenziato in Figura \ref{fig:SectorEtaView} (destra), tra le wheel vi sono degli spazi in cui non sono presenti rilevatori, pertanto vi sarà una minore densità di eventi rilevati in queste zone. Inoltre per $|\eta| > 1$ ci si trova nella regione di overlap, dove le stubs delle schede TwinMux potrebbero essere inviate ai rilevatori OMTF piuttosto che a quelli BMTF, e quindi non contenuti nei dati a disposizione.

Un'altra delle informazioni ricavate dall'algoritmo kBMTF è il momento del muone; in particolare vengono calcolati dal sistema due diversi tipi di momento trasverso: un momento trasverso \textit{unconstrained}, $p_T^u$ e uno \textit{vertex-constrained}, $p_T$. In Sezione \ref{sec:KalmannFilter} è stato detto che il Kalman filter esegue il tracking del muone a partire dalle stubs delle stazioni più esterne verso quelle interne: supponendo di porre nell'origine la posizione di collisione del fascio di protoni, si può considerare questa coordinata come punto finale della propagazione dell'algoritmo di Kalman, introducendo quindi un grado di libertà aggiuntivo, ottenendo una maggiore precisione: in questa ipotesi il kBMTF calcola il momento \textit{vertex-constrained}. \newline
Questa ipotesi non è però corretta, vi sono fluttuazioni non note sul punto di collisione del fascio di protoni e per tanto si può decidere di non considerare l'origine come un grado di libertà aggiuntivo dell'algoritmo. In tal caso il Kalman filter calcolerà il momento \textit{unconstrained}. 
Figura \ref{fig:BMTFPt} mostra la distribuzione del momento vertex-constrained e unconstrained dei candidati muoni.


\section{Confronto tra Global Muon Trigger e kBMTF}
In questa sezione verrà invece effettuato un confronto tra i muoni ottenuti dal sistema di Scouting dal Global Muon Trigger e i candidati muoni emulati, validandone la conformità assicurandosi che questi ultimi siano concordi con quelli ricavati dagli algoritmi implementati via hardware. 

Il criterio che utilizziamo per affermare che vi è un \textit{match} tra un candidato muone emulato e un muone del GMT del sistema di Data Scouting è studiare la distanza tra due muoni dello stesso BX nel piano $\phi - \eta$, ovvero:
\begin{equation}
  \label{eq:DeltaR}
  \Delta R = \sqrt{(\Delta \phi)^2 + (\Delta \eta)^2}
\end{equation}

Dove $\Delta \phi = \phi_{BMTF} - \phi_{GMT}$ e $\Delta \eta = \eta_{BMTF} - \eta_{GMT}$. Definiamo un buon criterio di match la condizione $\Delta R < 0.4$, nonostante per analisi più accurate il match sarebbe stabilito da $\Delta R < 0.1$.

Per verificare il match tra un muone del GMT e un candidato muone emulato del BMTF è stato prima di tutto isolato un muone del GMT in un determinato BX, calcolando poi la distanza nel piano nel piano $\phi - \eta$ rispetto ai candidati muoni del BMTF nello stesso BX, e quindi applicando la condizione di match. Se nel corrispettivo BX vi sono più muoni del BMTF viene scelto il candidato muone che minimizza il $\Delta R$.


\begin{figure}[t]
  \centering
  \begin{minipage}[b]{0.51\textwidth}
    \centering
    \includegraphics[width=\textwidth]{../Immagini/DeltaR.pdf} 
    \end{minipage}
    \hfill 
    \begin{minipage}[b]{0.48\textwidth}
      \centering
      \includegraphics[width=\textwidth]{../Immagini/BMTF_PtPtu.pdf} 
    \end{minipage}
    \caption{Distribuzione della distanza $\Delta R$ degli eventi con match (destra)}
  \label{fig:DeltaR}
\end{figure}

In Figura \ref{fig:DeltaR} viene mostrata la distribuzione delle distanze nel piano $\phi - \eta$. Come ci si aspetta la maggioranza degli eventi presenta una distanza $\Delta R = 0$




