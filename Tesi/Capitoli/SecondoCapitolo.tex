\chapter{Proprietà dei candidati muoni}
\label{cap:SecondoCapitolo}
%\let\cleardoublepage\clearpage

Nella Sezione \ref{sec:SistemaDiTrigger} si è discusso nel dettaglio del sistema di Trigger Level 1 di CMS e di come questo giochi un ruolo fondamentale nella selezione di eventi interessanti a seguito della collisione tra protoni, riducendo il volume di dati da gestire. In particolare è stato detto che le informazioni provenienti dai detector locali del sistema muonico nella regione di barrel e nella regione di overlap vengono combinate e processate dalle schede \textbf{TwinMux}, generando \textit{superprimitives}, o stubs, con una risoluzione temporale e spaziale elevata e che queste vengono poi utilizzate dal sistema di tracking del barrel (BMTF) o, a seconda della regione $\eta$ in cui i segnali vengono rilevati, dell'overlap (OMTF) per ricostruire la traccia del candidato muone.

Inoltre nella Sezione \ref{sec:DataScouting} si è introdotto il sistema di Data Scouting impiegato durante la Run 3 nel L1T che permette di analizzare, seppur con una minore risoluzione, eventi unbiased che verrebbero potenzialmente rigettati dal sistema di Trigger. In questo capitolo si procederà con lo studio e la validazione dei dati ottenuti dal sistema di Data Scouting, che raccoglie informazioni impiegando schede di acquisizione in vari step della catena di Trigger, in particolare a livello degli input del Barrel Muon Track Finder, del Global Muon Trigger e del Global Trigger.


\section{Validazione superprimitives}
\label{sec:Superprimitives}

Tra la Run 1 e la Run 2, sono state introdotte in CMS le schede TwinMux nella catena di Trigger: queste hanno il compito di generare superprimitives (da questo momento in poi \textit{stubs}) a partire dai segnali derivanti dai rilevatori Drift Tubes e Resistive Plate Chambers. \newline
Nel dettaglio i DT forniscono informazioni sulla posizione, direzione, qualità e bunch crossing ad un rate di 480Mb/s, mentre le RPC sulla posizione e sul BX con un rate di 1.6Gb/s \cite{CERNsummerSchool}.\newline
In particolare sono applicati algoritmi di clustering ai dati in input alle schede TwinMux, convertendo i segnali spaziali delle RPC nelle coordinate dei DT. Se i segnali provenienti dai due rilevatori sono compatibili, ovvero se $\Delta \phi < 15$ \si{mrad} tra i segmenti generati dai DT e i cluster delle RPC, allora vengono combinati.
Il sistema TwinMux può costruire fino a 2 stubs per stazione, che vengono poi inviate al BMTF e, in base alla posizione nel piano della pseudorapidità, all' OMTF. \newline
È inoltre funzione delle schede TwinMux assegnare dei \textit{parametri} a ciascuna stub generata: questi riguardano la posizione spaziale dei segnali in CMS, l'angolo di curvatura relativo alla stazione e la qualità della stub. I parametri sono riassunti nella Tabella \ref{table:StubsParam}. A partire da queste informazioni il BMTF ricostruisce la traccia del muone, determinando la traiettoria della particella. 

Questa sezione è il primo passo alla validazione del sistema di Data Scouting introdotto con la Run 3 nel L1T: vengono studiate le stubs, ovvero i segnali di ingresso al BMTF. In particolare verranno utilizzati 5 minuti e 22 secondi di presa dati, corrispondenti a circa $4 \times 10^{7}$ stubs. 

Come illustrato nella Sezione \ref{sec:LHC} in LHC circolano fasci di protoni che collidono nei punti di interazione dove sono presenti i principali esperimenti. Per ottimizzare e massimizzare le collisioni nei punti di interazione i pacchetti di protoni vengono disposti in modo specifico, facendo in modo che ogni fascio possa contenere al massimo 2808 pacchetti di protoni, nonostante ci siano 3564 posizioni disponibili lungo l'anello di LHC. La disposizione dei pacchetti di protoni nei fasci è chiamata \textit{filling scheme} e generalmente questa può variare in base all'esperimento. Il filling scheme utilizzato durante la presa dati è mostrato in Figura \ref{fig:Stubs1}, formato da 39 gruppi da 72 pacchetti di protoni a distanza 25ns l'uno dall'altro\cite{Bailey}. \newline
Si nota in Figura \ref{fig:Stubs1} come il rate di stubs non è mai nullo e vi sia sempre un fondo costante di eventi: questo fenomeno è da attribuire alla presenza di eventi di background come muoni cosmici, che vengono rilevati dai detector DT e RPC generando stubs. 



 \begin{figure}[t]
   \centering
   \begin{minipage}[b]{0.49\textwidth}
       \centering
       \includegraphics[width=\textwidth]{../Immagini/StubsBXnumber.pdf} 
     \end{minipage}
     \hfill 
     \begin{minipage}[b]{0.49\textwidth}
       \centering
       \includegraphics[width=\textwidth]{../Immagini/StubsBXnumberZoom.pdf} 
   \end{minipage}
   \caption{Filling scheme per la Run 3: disposizione dei pacchetti di protoni nelle 3564 possibili posizioni }
   \label{fig:Stubs1}
 \end{figure}


Viene studiata la distribuzione di stubs per bunch crossing (BX) e per Orbita (orbit). Ricordando che il BX è l'unità di misura che definisce l'intervallo temporale tra le collisioni, in Figura \ref{fig:StubsMolteplicity} a sinistra viene mostrato il numero di stubs raccolte per BX, ovvero in 25ns. È importante sottolineare che la maggior parte dei BX non contiene nessuna stubs, anche se questa informazione non è mostrata nell'immagine. Inoltre vi sono eventi che presentano un numero di stub raccolte per BX maggiore di 13, ma la frequenza di questi eventi è minore di 0.1 al secondo e per tanto non sono mostrati.\newline
Sempre in Figura \ref{fig:StubsMolteplicity} a destra, è rappresentato il numero di stubs raccolte in ogni orbita, ovvero un ciclo completo di LHC. Ogni orbita è formata da 3564 BX, ovvero circa 89 $\mu$s. Rappresentato su scala logaritmica, la distribuzione di stubs per orbita segue una distribuzione gaussiana simmetrica, centrata su 230 stubs. 

 \begin{figure}[t]
  \centering
  \begin{minipage}[b]{0.49\textwidth}
      \centering
      \includegraphics[width=\textwidth]{../Immagini/StubsMolteplicity.pdf} 
    \end{minipage}
    \hfill 
    \begin{minipage}[b]{0.49\textwidth}
      \centering
      \includegraphics[width=\textwidth]{../Immagini/StubsPerOrbit.pdf} 
  \end{minipage}
  \caption{Stubs multiplicity per BX (sinistra), Stubs multiplicity per orbit (destra)}
  \label{fig:StubsMolteplicity}
\end{figure}



Infine viene rappresentata la distribuzione spaziale di stubs nel corpo di CMS in Figura \ref{fig:StubsInCMS} per verificare se vi sono anormalità nel sistema di acquisizione dati. Come specificato in Sezione \ref{sec:CMSDescrizione} e Figura \ref{fig:SectorEtaView}, CMS è formato da cinque ruote (\textit{wheel}), ognuna suddivisa in dodici settori (\textit{sector}) a loro volta contenenti quattro stazioni (\textit{station}). Convenzionalmente le wheel sono nominate da -2 a 2, dove 0 rappresenta la ruota in corrispondenza al punto di collisione dei protoni (origine degli assi), i sector sono nominati da 0 a 11 e le stazioni da 1 a 4. La Figura \ref{fig:StubsInCMS} mostra quindi la distribuzione spaziale delle stubs: nel grafico di sinistra notiamo una più alta densità di stubs nelle wheel $\pm$ 1, mentre una minore porzione di eventi vengono raccolti nelle wheel $\pm$ 2. Questo è dovuto al fatto che in corrispondenza di queste wheel ci troviamo nella regione ad alta pseudorapidità e per tanto vicini alla regione di \textit{overlap}: gli eventi in questa regione è probabile che vengano raccolti dal OMTF e dunque parzialmente schermati al BMTF. Nella figura di destra invece notiamo una maggiore densità in corrispondenza della prima stazione per le wheel $\pm$ 1, mentre vi è una completa assenza di eventi nella stessa stazione per le wheel $\pm$ 2. Questo difetto è attualmente noto e si verifica poiché stubs in questa stazione vengono raccolte solamente dal sistema di tracking dell'overlap, non venendo lette dal BMTF.

\renewcommand{\arraystretch}{1.25} % Aumenta lo spazio tra le righe
\begin{table}[t]
 \caption{Descrizione dei parametri assegnati dal TwinMux a ciascuna stub}
 \label{table:StubsParam}
  \centering
  \begin{tabular}{cccl}
  \toprule
  \textbf{Parameter} & \textbf{Bits} & \textbf{Range} & \textbf{Description} \\ \hline
  $\phi$            & 12  & [$-2048$, 2047]   & Relative position of a segment inside a sector \\ \hline
  $\phi_b$          & 10  & [$-512$, 511]     & Bending angle \\ \hline
  \textit{quality}  & 3   & [0, 7]            & Number of superlayers used to construct the stub \\ \hline
  $\eta$ hits       & 7   & "pattern"         & \begin{tabular}[c]{@{}l@{}} Each bit corresponds to one chamber area \\ 0 : no hit (less than 3 SL hits) \\ 1 : hit (3 or 4 SL hits) \end{tabular} \\ \hline
  $\eta$ quality    & 7   & "pattern"         & \begin{tabular}[c]{@{}l@{}} Each bit corresponds to one chamber area \\ 0 : 3 SL hits \\ 1 : 4 SL hits \end{tabular} \\ 
  \bottomrule
  \end{tabular}
  
\end{table}


  \begin{figure}[t]
    \centering
    \begin{minipage}[b]{0.49\textwidth}
        \centering
        \includegraphics[width=\textwidth]{../Immagini/StubsSectorWheel.pdf} 
      \end{minipage}
      \hfill 
      \begin{minipage}[b]{0.49\textwidth}
        \centering
        \includegraphics[width=\textwidth]{../Immagini/StubsStationWheel.pdf} 
      \end{minipage}
      \caption{Distribuzione stubs nel corpo di CMS come grafico bidimensionale}
    \label{fig:StubsInCMS}
  \end{figure}





\section{Validazione candidati muoni del BMTF}
\label{sec:BMTF}

Il passo successivo nella catena di identificazione dei muoni viene eseguito dal sistema di tracking che ricostruisce la traccia del \textit{candidato} muone a partire dalle informazioni inoltrate dalle schede TwinMux. In questo studio verranno considerate solo le tracce di muoni nella zona di barrel, pertanto l'analisi verrà eseguita sulle informazioni nella regione di pseudorapidità $|\eta| < 1.2$: come già specificato nella Sezione \ref{sec:SistemaDiTrigger} il Barrel Muon Track Finder è il sistema che si occupa della ricostruzione delle tracce in questa regione.

Il BMTF implementa specifici algoritmi in hardware che permettono di convertire efficacemente le stubs generate delle schede TwinMux in candidati muoni. Dalla Tabella \ref{table:StubsParam}, ogni stub possiede informazioni spaziali e di qualità assegnate dal sistema TwinMux: in particolare 12 bit sono riservati alle coordinate $\phi$, 10 bit all'angolo di inclinazione $\phi_b$ e 3 bit alla qualità della stub. Inoltre sono riservati 7 bit per i segnali rilevati per regione di $\eta$ e 7 per la qualità di $\eta$. Di seguito verrà discusso del sistema di tracking che utilizzano le schede BMTF per ricostruire in hardware la traccia del candidato muone e del sistema di \textit{emulazione} di CMS via software che permette, attraverso specifici parametri, di emulare i risultati ottenuti in hardware.

\subsection{Ricostruzione hardware dei muoni}
\label{sec:KalmannFilter}

La ricostruzione delle tracce di muoni in hardware a CMS si sviluppa principalmente in tre step: 
\begin{itemize}
  \item Propagazione dei parametri delle tracce ricevute dal sistema di TwinMux;
  \item Ricerca di segnali vicini compatibili con la traccia;
  \item Aggiornamento dei parametri di ricostruzione utilizzando \textit{Kalman filter} \cite{Summers:2728522}
\end{itemize}

Questo algoritmo sviluppato da R.E. Kalman nel 1960, consente di descrivere ricorsivamente lo stato di un sistema dinamico, minimizzando il rumore e l'incertezza delle misurazioni: questo lo rende un'ottima scelta negli algoritmi di ricostruzione, in particolare a CMS \cite{welch1995introduction}. L'algoritmo di ricostruzione che usa Kalman filter è in funzione in CMS dal 2018 ed è stato rinominato kBMTF. 

Nel dettaglio, il vettore di stato $x_n = (k, \phi, \phi_b)$ rappresenta i parametri della traccia ad ogni stazione; definiamo $k = q/p_T$, dove $q$ è la carica del muone in esame e $p_T$ il momento trasverso. Il sistema di tracking utilizzato con il Kalman filter considera inizialmente le stubs nelle stazioni più esterne, propagando la traccia verso l'interno del CMS usando la seguente equazione:

\begin{equation}
  \label{eq:KalmanFilterEq}
    x_n = F x_{n-1}
\end{equation}

Dove $F$ è una matrice che descrive la geometria e l'intensità del campo magnetico del detector in ogni stazione.
Dopo di che la stub più vicina viene identificata, $z_n = (\phi, \phi_b)$ e quindi il vettore di stato viene aggiornato, $\bar{x}_n$ e viene calcolato il residuo $r_n$ tra il vettore di stato $x_n$ e la stub $z_n$: 

\begin{equation}
  \label{eq:KalmanFilterEq2}
    r_n = z_n - Hx_n = \left( \begin{matrix}
      \phi \\
      \phi_b
    \end{matrix}\right)_n - \left( \begin{matrix}
      0 & 1 & 0 \\ 
      0 & 0 & 1
    \end{matrix}\right)
    \left(\begin{matrix}
      k \\
      \phi \\
      \phi_b
    \end{matrix}\right)_n
\end{equation}

\begin{equation}
  \label{eq:KalmanFilterEq3}
    \bar{x}_n = x_n + Gr_n
\end{equation}

Dove $G$ è la matrice di Gain (Kalman Gain Matrix) e dipende dalle incertezze della previsione della misurazione: se la misurazione ha basso rumore allora il guadagno sarà alta, mentre se il rumore è alto il guadagno sarà basso. Senza entrare nel dettaglio della costruzione della matrice di Gain, si intuisce che questa dipende dalla matrice di covarianza alla iterazione $n$. Dando solo la definizione, la matrice di covarianza è la seguente:

\begin{equation}
  \label{eq:covariance}
  P_n = FP_{n-1}F^T + Q(k, x/X_0)
\end{equation}

Se nello stesso BX ci sono più tracce di muoni il Kalman filter considera solamente la traccia con $\chi^2$ minore, minimizzando quindi l'errore quadratico medio \cite{Summers:2728522}

\subsection{Ricostruzione software dei muoni}
\label{sec:KalmanFilterSoftware}

Il Kalman filter è quindi un algoritmo che viene applicato per la ricostruzione di muoni nella catena di Trigger e per tanto deve soddisfare i requisiti di latenza richiesti dal Trigger stesso, oltre che i limiti di banda di archiviazione. Questo non permette il passaggio di dati a risoluzione massima di alcune quantità forzando il sistema a troncare, o approssimare, le informazioni relative ai candidati muoni. Il sistema di emulazione del kBMTF permette quindi una analisi più fine rispetto alla analisi online, evitando le limitazioni introdotte dal sistema di Trigger. Il software utilizzato per l'emulazione del Kalman filter permette di ricavare informazioni circa il momento trasverso $p_T$, angolo azimutale $\phi$ e pseudorapidità $\eta$, e qualità, oltre che altri parametri.

In questa sezione verranno studiati i grafici di occupazione e di molteplicità dei candidati muoni emulati dal software di CMS, verificando quindi le differenze rispetto ai grafici della Sezione \ref{sec:Superprimitives}.



\begin{figure}[t]
  \centering
  \begin{minipage}[b]{0.49\textwidth}
      \centering
      \includegraphics[width=\textwidth]{../Immagini/BMTF.pdf} 
    \end{minipage}
    \hfill 
    \begin{minipage}[b]{0.49\textwidth}
      \centering
      \includegraphics[width=\textwidth]{../Immagini/BMTF_zoom.pdf} 
    \end{minipage}
    \caption{Filling Scheme dei candidati muoni emulati in software }
  \label{fig:BMTFMuons}
\end{figure}

\begin{figure}[t]
  \centering
  \begin{minipage}[b]{0.49\textwidth}
      \centering
      \includegraphics[width=\textwidth]{../Immagini/BMTF_Molteplicity.pdf} 
    \end{minipage}
    \hfill 
    \begin{minipage}[b]{0.49\textwidth}
      \centering
      \includegraphics[width=\textwidth]{../Immagini/BMTF_orbit.pdf} 
    \end{minipage}
    \caption{Molteplicità dei candidati muoni emulati in software per BX (sinistra) e per orbita (destra)}
  \label{fig:BMTFMolteplicity}
\end{figure}

In Figura \ref{fig:BMTFMuons} viene mostrato il filling scheme nei 3564 Bunch Crossing. Notiamo un minor rate di eventi rispetto a quello mostrato in Figura \ref{fig:Stubs1} in quanto questo filling scheme rappresenta i \textit{candidati muoni}, costruiti a partire dalle stubs delle schede TwinMux e non tutte le stubs sono associate a muoni dal sistema di Tracking. Ciò è reso visibile nella Figura \ref{fig:BMTFMolteplicity} sulla sinistra: la maggior parte dei BX presenta un numero nullo di candidati muoni. Inoltre, il numero di BX contenenti un numero non nullo di muoni sembra decrescere esponenzialmente all’aumentare della molteplicità, come d'altronde succedeva per le stubs in Figura \ref{fig:StubsMolteplicity}.
%Particolare attenzione deve essere posta anche sul numero di eventi di fondo: rispetto a quanto osservato nella sezione precedente, il rate di questi eventi è notevolmente inferiore. 

Sempre in Figura \ref{fig:BMTFMolteplicity}, sulla destra è mostrata la distribuzione in scala logaritmica dei candidati muoni per orbita: notiamo che la distribuzione è sempre gaussiana, centrata a circa 30 muoni ma, rispetto a quanto mostrato nella sezione precedente, vi è una asimmetria della curva che la rende inclinata positivamente, mentre la distribuzione in Figura \ref{fig:StubsMolteplicity} è simmetrica.

\begin{figure}[t]
  \centering
  \begin{minipage}[b]{0.51\textwidth}
    \centering
    \includegraphics[width=\textwidth]{../Immagini/BMTFPhiEta.pdf} 
    \end{minipage}
    \hfill 
    \begin{minipage}[b]{0.48\textwidth}
      \centering
      \includegraphics[width=\textwidth]{../Immagini/BMTF_PtPtu.pdf} 
    \end{minipage}
    \caption{Distribuzione dei candidati muoni come grafico bidimensionale in $\eta$ e $\phi$ (sinistra), distribuzione del momento trasverso unconstrained e vertex-constrained (destra)}
  \label{fig:BMTFPt}
\end{figure}

Il candidato muone ricostruito a partire dall'algoritmo kBMTF, come anche quello emulato in software, possiede informazioni sulla posizione spaziale nelle coordinate di CMS: viene quindi assegnato un valore $\eta$ ed un valore $\phi$ ad ogni candidato muone. In Figura \ref{fig:BMTFPt}, sulla sinistra, è rappresentata la distribuzione bidimensionale di eventi nelle due coordinate: a partire da queste informazioni è possibile verificare la conformità dei rilevatori in $\phi$ ed $\eta$. È evidente che lungo il piano $\eta = k$, con $k$ una costante, non vi siano particolarità: come ci si aspetta essendo il sistema a simmetria cilindrica la distribuzione degli eventi è costante. Si nota però una notevole asimmetria lungo $\phi = k$: ciò è dovuto alla geometria del rilevatore CMS: come evidenziato in Figura \ref{fig:SectorEtaView} (destra), tra le wheel vi sono degli spazi in cui non sono presenti rilevatori, pertanto vi sarà una minore densità di eventi rilevati in queste zone. Inoltre per $|\eta| > 1$ ci si trova nella regione di overlap, dove le stubs delle schede TwinMux potrebbero essere inviate ai rilevatori OMTF piuttosto che a quelli BMTF, e quindi non contenuti nei dati a disposizione.

Un'altra delle informazioni ricavate dall'algoritmo kBMTF è il momento del muone. In particolare vengono calcolati dal Kalman filter due diversi tipi di momento trasverso: un momento trasverso \textit{unconstrained}, $p_T^u$ e uno \textit{vertex-constrained}, $p_T$. In Sezione \ref{sec:KalmannFilter} è stato detto che il Kalman filter esegue il tracking del muone a partire dalle stubs delle stazioni più esterne verso quelle interne: supponendo di porre nell'origine la posizione di collisione del fascio di protoni, si può considerare questa coordinata come punto finale della propagazione dell'algoritmo di Kalman, introducendo quindi un grado di libertà aggiuntivo, ottenendo una maggiore precisione di ricostruzione: in questa ipotesi il kBMTF calcola il momento \textit{vertex-constrained}. \newline
Questa ipotesi non è però corretta, vi sono fluttuazioni non note sul punto di collisione del fascio di protoni e per tanto si può decidere di non considerare l'origine come un grado di libertà aggiuntivo dell'algoritmo. In tal caso il Kalman filter calcolerà il momento \textit{unconstrained}. 
Figura \ref{fig:BMTFPt} mostra la distribuzione del momento vertex-constrained e unconstrained dei candidati muoni. È interessante notare un maggiore riempimento dei bin ad alto $p_T$ e $p_T^u$: questo è dovuto al fatto che tutti gli eventi con un momento maggiore di 255GeV/c vengono racchiusi in un bin di \textit{oveflow} che raccoglie in un unico bin tutti gli eventi oltre questa soglia massima.


\section{Confronto tra Global Muon Trigger e kBMTF}
\label{sec:Confronto}

In questa sezione verrà effettuato il confronto tra i candidati muoni \textit{emulati} del BMTF, che presentano quindi una risoluzione elevata, e i muoni ottenuti dalla catena di Trigger dal sistema di Data Scouting del GMT, validando la corrispondenza tra essi a partire dalle loro caratteristiche spaziali e verificando che queste rientrino in determinati criteri di match. 

Il criterio utilizzato per verificare il \textit{match} tra un candidato muone emulato e un muone del GMT è studiare la distanza dei due nel piano $\phi - \eta$, ovvero:

\begin{equation}
  \label{eq:DeltaR}
  \Delta R = \sqrt{(\Delta \phi)^2 + (\Delta \eta)^2}
\end{equation}

Dove $\Delta \phi = \phi_{\mathrm{BMTF}} - \phi_{\mathrm{GMT}}$ e $\Delta \eta = \eta_{\mathrm{BMTF}} - \eta_{\mathrm{GMT}}$. Un buon criterio di match è definito dalla condizione $\Delta R < 0.4$, nonostante per analisi più accurate il match sarebbe stabilito da $\Delta R < 0.1$.

In particolare è possibile applicare la condizione di match in due modi diversi. Il primo metodo consiste nel fissare un candidato muone del BMTF in un determinato BX, calcolando la distanza $\Delta R$ tra esso e tutti i muoni del GMT nel corrispondente BX, applicando la condizione di match e scegliendo la coppia che minimizza la distanza. \newline
Il secondo metodo invece verifica il match isolando un muone del GMT e calcolando la distanza dai corrispondenti candidati muoni emulati del BMTF, applicando la condizione di match in modo che la distanza della coppia venga minimizzata.


%dire della duplicazione????????


\begin{figure}[t]
  \centering
  \begin{minipage}[b]{0.49\textwidth}
    \centering
    \includegraphics[width=\textwidth]{../Immagini/DeltaR.pdf} 
    \end{minipage}
    \hfill 
    \begin{minipage}[b]{0.46\textwidth}
      \centering
      \includegraphics[width=\textwidth]{../Immagini/PtMatchingEfficiency.pdf} 
    \end{minipage}
    \caption{distribuzione della distanza nel piano $\Delta \phi - \Delta \eta$ degli eventi del secondo metodo con match (sinistra), Matching efficiency in funzione del vertex constrained $p_T$ per il primo metodo (BMTF) e il secondo (GMT) (destra)}
  \label{fig:DeltaR}
\end{figure}


Applicando entrambi i metodi, la percentuale di eventi senza match per il primo metodo è 5.58\%, molto più elevata rispetto al secondo metodo che corrisponde allo 0.23\%. La spiegazione della discrepanza tra i due metodi è che, ricordando che per costruzione dei dati in questo studio vengono considerate solamente le informazioni provenienti dalla regione di barrel, nel primo caso solamente 108 candidati muoni vengono inviati al GMT, come sottolineato nella Sezione \ref{sec:SistemaDiTrigger}, per tanto non tutti i candidati muoni del BMTF sono passati dal GMT. Ovvero in un determinato BX il sistema di Trigger potrebbe aver scartato i candidati muoni del BMTF preferendo quelli del OMTF o EMTF, generando un evento senza match nel primo metodo. \newline
Al contrario nel secondo metodo, tutti i muoni del GMT provengono dal BMTF per costruzione, dunque ci si aspetta una percentuale di eventi senza match molto bassa, come effettivamente è verificato.

In Figura \ref{fig:DeltaR} a sinistra viene mostrata la distribuzione bidimensionale delle distanze nel piano $\Delta \phi - \Delta \eta$ confrontando il match dei candidati muoni del BMTF con i muoni del GMT (secondo metodo). Notiamo che la distribuzione di questi eventi presenta un picco nell'origine del piano: ciò suggerisce un buon accordo per la maggior parte dei muoni del GMT e i candidati muoni emulati del BMTF.

%Confrontando quanto ottenuto in questo studio con i dati del 2023 \cite{CERNsummerSchool} notiamo un netto miglioramento nella distribuzione degli eventi, con un maggiore densità nell'origine del piano, oltre che una distribuzione globale più ristretta verso l'origine. Inoltre si nota come non sia più presente la maggiore densità di eventi in $\Delta \phi = 0$.


\begin{figure}[t]
  \centering
  \begin{minipage}{0.32\textwidth}
      \centering
      \includegraphics[width=\textwidth]{../Immagini/QualLowHighPt.pdf}
  \end{minipage}%
  \hfill % Spaziatura tra immagini
  \begin{minipage}{0.32\textwidth}
      \centering
      \includegraphics[width=\textwidth]{../Immagini/EtaLowHighPt.pdf}
  \end{minipage}%
  \hfill
  \begin{minipage}{0.32\textwidth}
      \centering
      \includegraphics[width=\textwidth]{../Immagini/GMTQualLowHighPt.pdf}
  \end{minipage}
  \caption{Distribuzione della qualità per candidati muoni del BMTF nell'intervallo $p_T \in [0, 5.5]$ GeV/c e $p_T \in [100, 255]$ GeV/c (sinistra), distribuzione della qualità per candidati muoni del BMTF nell'intervallo $p_T \in [0, 5.5]$ GeV/c e $p_T \in [100, 255]$ GeV/c (centro), distribuzione di qualità per muoni del GMT negli stessi intervalli (destra)}
  \label{fig:MatchingEfficiecyLowPt}
\end{figure}





\subsection{Studio della matching efficiency}
\label{sec:MatchingEfficiency}

È possibile estendere il confronto tra i due metodi verificando la \textit{matching efficiency} dei candidati muoni del BMTF e del GMT in funzione del momento vertex-constrained. Estraendo quindi l'informazione sul momento di ogni muone con match per entrambi i metodi è possibile ottenere il grafico in Figura \ref{fig:DeltaR} sulla destra, che mostra la matching efficiency in funzione del momento vertex-constrained. 
Questa corrisponde, per entrambi i metodi, al rapporto tra muoni con match e muoni totali in funzione di $p_T$, dove i muoni con match sono un sottoinsieme dei muoni totali. Questo è pertanto un processo binomiale: per calcolare intervalli di confidenza di distribuzioni di questo tipo in fisica delle particelle viene spesso utilizzato il metodo di \textit{Clopper-Pearson} in quanto per dati discreti e campioni piccoli è più accurato. La larghezza dell’intervallo di confidenza fornita dal metodo Clopper-Pearson può essere vista come un’indicazione dell’incertezza totale sulla stima di match.


Notiamo che la matching efficiency dei candidati muoni del BMTF (primo metodo) presenta una decrescita esponenziale a momenti $p_T$ molto bassi, seguiti da una crescita logaritmica per momenti maggiori. Questo fenomeno è spiegato dal modo in cui il sistema di Trigger decide di selezionare un evento del BMTF, inviandolo al GMT: muoni con un momento molto elevato sono indice di fenomeni interessanti in quanto spesso sono firma a processi di produzione di particelle massive, come bosoni di Higgs, e pertanto ad essi è frequentemente associata una qualità maggiore. Ciò però non è sempre vero in quanto un muone con un momento ($p$) troppo elevato è difficilmente ricostruibile dal sistema di tracking, in quanto la carica ($q$) dello stesso viene dedotta a partire dalla direzione di curvatura del muone nel campo magnetico (B) secondo la relazione di Lamor:

\begin{equation}
  \label{eq:Lamor}
  r = \frac{p}{qB}
\end{equation}

e quindi un momento elevato implica un raggio di curvatura elevato che è difficilmente ricostruibile dal sistema di tracking. È quindi possibile che muoni con un momento molto elevato non vengano inviati al GMT, producendo un evento senza match.  
Allo stesso modo eventi con un momento ridotto sono ricostruiti molto bene a partire dall'Equazione \ref{eq:Lamor}, ma sono generalmente indice di fenomeni poco interessanti. Il comportamento della matching efficiency in queste regioni di momento è presumibilmente indice che il sistema di Trigger favorisce i muoni derivanti dal sistema di tracking della regione dell'Overlap o dell'Endcap piuttosto che quelli del Barrel.

Al contrario, come ci si aspetta, la matching efficiency per muoni del GMT (secondo metodo) risulta essere quasi costante e vicina ad 1, in quanto quasi tutti i muoni del GMT hanno un match con un candidato muone del BMTF, come illustrato in Sezione \ref{sec:Confronto}. 


È possibile caratterizzare ulteriormente la matching efficiency in funzione del momento, studiando la distribuzione di qualità e di $\eta$ dei muoni in determinati intervalli di momento. Figura \ref{fig:MatchingEfficiecyLowPt} a sinistra rappresenta le distribuzioni di qualità degli eventi negli intervalli di momento $p_T^l \in [0, 5.5]$ GeV/c e $p_T^h \in [100, 255]$ GeV/c. Si nota come le due distribuzioni non corrispondano e che $p_T^h$ presenti una maggiore probabilità di eventi a qualità elevate. In particolare, definendo il rapporto delle due distribuzioni di qualità, si nota come questo valga 9.5 $\pm$ 2.7 a qualità 15. Ciò significa che se un muone ha qualità alta è mediamente quasi 10 volte più probabile che abbia momento elevato. L'incertezza del rapporto è stata calcolata propagando gli errori delle due distribuzioni binomiali.

\begin{figure}[t]
  \centering
  \begin{minipage}[b]{0.47\textwidth}
    \centering
    \includegraphics[width=\textwidth]{../Immagini/EtaHighPt.pdf} 
    \end{minipage}
    \hfill 
    \begin{minipage}[b]{0.51\textwidth}
      \centering
      \includegraphics[width=\textwidth]{../Immagini/2DEtaHighPt.pdf}
    \end{minipage}
    \caption{Distribuzione dei muoni del GMT (primo metodo) in $\eta$  nell'intervallo $p_T \in [0, 255]$ GeV/c e $p_T \in [100, 255]$ GeV/c (sinistra), distribuzione bidimensionale dei muoni del GMT in $\eta$ e $p_T$ (destra)}
  \label{fig:MatchingEfficiecyPt2}
\end{figure}

Rimanendo sempre nello stesso intervallo di momento è possibile studiare anche la distribuzione in $\eta$ dei muoni del BMTF, come illustrato in Figura \ref{fig:MatchingEfficiecyLowPt} in centro. Anche in questo caso le distribuzioni nei due intervalli non coincidono ed in particolare, studiando nuovamente il rapporto si nota una maggiore densità di probabilità di eventi a basso momento ad nella regione ad elevato $\eta$ e viceversa una maggiore densità di eventi ad elevato momento nella regione a basso $\eta$. \newline
Questo implica che un muone con $p_T$ ridotto ha maggiori probabilità di trovarsi in una regione a elevato valore di $\eta$, mentre un muone con $p_T$ elevato è più probabile che si trovi in una regione a basso valore di $\eta$.




Fino ad ora sono stati caratterizzati i muoni del BMTF relativi alla matching efficiency in Figura \ref{fig:DeltaR}. Notiamo però sempre in Figura \ref{fig:DeltaR} a sinistra che la matching efficiency dei muoni del GMT non è veramente costante in funzione del momento ed in particolare la maggior parte degli eventi senza match risiede nella regione a $p_T$ elevato. È quindi possibile cercare di caratterizzare questo fenomeno verificando, come fatto per i muoni del BMTF, la distribuzione della qualità e di $\eta$ negli stessi $p_T^h$ e $p_T^l$. Questo viene illustrato in Figura \ref{fig:MatchingEfficiecyLowPt} sulla destra dove si nota, come succedeva per i muoni del BMTF, una maggiore densità di eventi ad elevato momento alla qualità 15; in questo caso il rapporto tra le due distribuzioni a qualità elevata risulta essere 25 $\pm$ 6, ovvero un muone con qualità elevata è mediamente 25 volte più probabile avere un momento molto elevato. \newline
Per indagare ulteriormente la diminuzione della matching efficiency relativa ai muoni del GMT si studia la densità di probabilità in $\eta$ dei muoni negli intervalli $p_T^h \in [100, 255]$ GeV/c e $p_t^a \in [0, 255]$ GeV/c. Questo viene illustrato in Figura \ref{fig:MatchingEfficiecyPt2} sulla sinistra dove si nota un comportamento simile a quanto visto per i muoni del BMTF con una maggiore densità di eventi a momento ridotto nelle regioni ad elevato valore di $\eta$. Si notano inoltre regioni con una maggiore densità di eventi a momento elevato, oltre che una densità nulla di questi muoni nella regione $|\eta| > 0.9$. La spiegazione di quest'ultimo fenomeno è intuibile in Figura \ref{fig:MatchingEfficiecyPt2} sulla destra, dove viene studiata la distribuzione di muoni con momento nell'intervallo $p_T^h$ in tutto il piano $\eta$. Qui si nota come nella regione $|\eta| > 0.9$ ci siano sempre meno eventi all'aumentare del momento: questo suggerisce che generalmente vi siano pochi muoni in questa regione in quanto possibilmente vengono favoriti dal sistema di Trigger gli eventi provenienti dai sistemi OMTF e EMTF. 
Ciò spiega almeno in parte la diminuzione della matching efficiency a momenti elevati per muoni del GMT.

\begin{figure}[t]
  \centering
  \begin{minipage}[b]{0.49\textwidth}
    \centering
    \includegraphics[width=\textwidth]{../Immagini/EtaMatchingEfficiency.pdf} 
    \end{minipage}
    \hfill 
    \begin{minipage}[b]{0.49\textwidth}
      \centering
      \includegraphics[width=\textwidth]{../Immagini/QualMatchingEfficiency.pdf}
    \end{minipage}
    \caption{Matching efficiency dei muoni del BMTF e del GMT in funzione di $\eta$ (sinistra) e della qualità (destra)}
  \label{fig:MatchingEfficiecyPt3}
\end{figure}


Viene infine studiata la matching efficiency dei candidati muoni del BMTF e del GMT in funzione di $\eta$ e della qualità, come illustrato in Figura \ref{fig:MatchingEfficiecyPt3}. Sulla sinistra in arancione è mostrata la matching efficiency in funzione di $\eta$ dei muoni del GMT con match nei muoni del BMTF: notiamo una distribuzione costante per regioni di $\eta$ positive, e una leggera diminuzione per regioni di $\eta$ negative. In blu è invece mostrata la matching efficiency di muoni del BMTF con match in muoni del GMT: notiamo qui una maggiore matching efficiency nelle regioni con $|\eta| > 0.9$ e una generalmente più bassa altrove. \newline
Questo fenomeno è difficilmente spiegabile con i dati a disposizione: potremmo supporre che il sistema di Trigger favorisca i muoni del BMTF nelle regioni ad elevato $\eta$, seppur questi eventi siano in numero minore come mostrato in Figura \ref{fig:BMTFPt} sulla destra. A rafforzare questa ipotesi è noto che i muoni rilevati dal BMTF hanno di base una qualità maggiore rispetto a quelli del OMTF e EMTF, ma senza informazioni su questi due sistemi di tracciamento non possiamo affermare se questa ipotesi è veramente corretta.

Sulla destra invece è mostrata la matching efficiency in funzione della qualità dei muoni: come ci si aspetta per i muoni del GMT questa è quasi costante, in particolare si nota una leggera crescita all'aumentare della qualità e ciò è spiegabile dal momento che muoni con una qualità maggiore è più probabile che vengano inviati al GMT, generando un match. Per i muoni del BMTF la matching efficiency sembra diminuire al crescere della qualità, ma anche questo fenomeno sembra difficilmente spiegabile con le informazioni a disposizione.



