\newenvironment{dedication}
    {
    \clearpage % Start on a new page
    \thispagestyle{empty} % Remove page numbers
    \vspace*{\stretch{1}} % Pushes text down towards the center of the page
    \begin{center}\begin{em}}
    {\end{em}\end{center}
    \vspace*{\stretch{2}} % Pushes text further down and balances it on the page
    \clearpage % Ends with a new page if desired
    }



\begin{dedication}
  Voglio dedicare questo mio primo vero traguardo a tutte le persone che mi hanno fatto stare bene in questi anni, standomi vicino, facendomi ridere e tenendomi compagnia. 

  Ringrazio i miei genitori, Chiara e Luigi, che mi hanno sempre incoraggiato a tenere la testa alta, ad andare avanti anche quando la strada era tortuosa e il tempo avverso. Nei momenti peggiori siete sempre stati al mio fianco, dimostrando tutto l'amore che dei genitori possono avere per un figlio. Grazie per aver creduto in me. Grazie mamma e papà.

  Ringrazio mia sorella Marianna, alla quale voglio un mondo di bene, che cerca sempre di farmi ridere raccontandomi di opere d'arte, libri o film. Anche se non mi interessano, riesci sempre a strapparmi un sorriso. Grazie perché so che sei sempre al mio fianco quando ne ho bisogno e so che posso sempre contare su di te, per una risata o per parole di conforto. Grazie Mary.

  Ringrazio anche i nonni, Franca ed Ezio, perché mi vogliono molto tanto bene e so che sono felici di questo mio traguardo. Passare le sere d'estate sul terrazzo, affacciati sul Lago a giocare a carte, a ridere e a raccontarsi storie sono i momenti che rimarranno sempre nel mio cuore. Ringrazio anche i nonni che non sono qui, ma che sono sempre nei miei pensieri, Maria e Benito. Sono sicuro che se potessero vedermi oggi sarebbero tanto orgogliosi di me, di quello che sono e di quello che diventerò. Grazie nonni.

  E infine ringrazio la mia famiglia, gli zii, i cugini 

  \vspace{2cm}


  Ringrazio i miei amici dell'università, a partire da Darius, Giorgio ed Ilias, le prime persone che ho conosciuto e che sono sempre state una costante nelle mie giornate, tra risate e disperazione. Assieme a voi però, Filippo e Giacomo, il nostro gruppo era davvero al completo. Vi ringrazio, perché assieme abbiamo passato momenti indimenticabili e quando penserò agli anni dell'università non ricorderò le difficoltà, ma le brevi pause in vostra compagnia tra una lezione e l'altra, le risate, i meme, i tergicristalli e gli sporadici spritz dopo gli esami. Grazie amici.

  Ringrazio anche tutti gli altri amici conosciuti all'università, perché tra chiacchiere e risate siete sempre riusciti a sollevarmi l'umore e alleggerire le giornate.

  Ringrazio anche i miei amici del Lago, quelli di casa. Laura, Chiara e Luca, sappiate che non vi ringrazio solo per questi ultimi tre anni in cui, seppur ci siamo visti poco, ci siamo visti sempre volentieri, ma anche per i cinque passati alle superiori. La vita da adulti ci ha separati, allontanandoci da quella piccola gabbia di matti, ma quando ci troviamo insieme tutti e quattro torno sui banchi di scuola, tra zaini girati, dizionari volanti e risate in laboratorio. Grazie perché quando penso a voi penso a tutto il divertimento abbiamo avuto assieme. Grazie anche a voi amici.
  
  
  Infine ringrazio la persona che mi ha accompagnato in questi mesi, che ha vissuto con i miei occhi le mie paure e le mie insicurezze, ma soprattutto la mia felicità, la mia gioia e i miei sorrisi. Grazie Noemi.
  
  \vspace{2cm}
  
  So che questo è il primo scalino della mia vetta, ma sappiate che durante la camminata non mi sono mai sentito solo con voi. Grazie per avermi accompagnato. 
\end{dedication}