\chapter{Introduzione}
\label{cap:Introduzione}

Negli ultimi decenni la fisica delle particelle ha compiuto grossi passi avanti nella ricerca dei costituenti fondamentali della materia, arrivando ad una più profonda comprensione delle interazioni che governano il mondo che ci circonda. Grazie ad esperimenti condotti negli acceleratori di particelle, come il Large Hadron Collider, LHC, al CERN, si sono ottenute conferme sperimentali del Modello Standard, arrivando al giorno d'oggi all'esplorazione di Nuova Fisica, ovvero la ricerca di conferme di teorie al di fuori della comprensione teorica attuale.

In questo contesto, i muoni giocano un ruolo cruciale nei principali esperimenti effettuati nel LHC, nello specifico a CMS: queste particelle, simili agli elettroni ma con una massa circa 200 volte maggiore, si distinguono per la capacità di attraversare strati di materiale del detector senza subire una significativa perdita di energia per unità di lunghezza. I muoni sono perciò largamente utilizzati per la rilevazione di fenomeni interessanti derivanti dalle collisioni di protoni in LHC: per questo motivo la rilevazione e la ricostruzione dei muoni è un processo di fondamentale importanza a CMS.\newline
In questo studio sono presentate le proprietà dei candidati muoni ottenuti direttamente dal sistema di Trigger di primo livello a CMS, noto come L1T. Il sistema di Trigger riceve eventi ad un rate di 40MHz derivanti dalle collisioni di protoni, e ha lo scopo di ridurre il volume di informazioni, mantenendo solamente quelle interessanti.
È discusso nel Capitolo \ref{cap:PrimoCapitolo} come la selezione di eventi fisici da parte del sistema di Trigger introduca un bias, ovvero un pregiudizio sui dati, dal momento che vengono mantenuti solamente gli eventi con caratteristiche compatibili con le leggi fisiche note, mascherando possibili informazioni che potrebbero portare a segni di fisica Oltre il Modello Standard. A questo scopo viene introdotto a CMS il sistema di Data Scouting nel L1T, ovvero un sistema che consente di raccogliere informazioni derivanti dalle collisioni direttamente dalla catena di Trigger, seppur con una minore risoluzione, lavorando parallelamente ad essa. Il sistema di Data Scouting nel L1T verrà implementato definitivamente a CMS con la \textit{Phase 2}, in contemporanea all'upgrade di LHC, nominato High-Luminosity LHC. In previsione della \textit{Phase 2}, durante la \textit{Run 3} a CMS è stato introdotto un sistema di Data Scouting apposito a livello del L1T che consente la validazione e la sperimentazione di nuovi algoritmi da implementare con l'upgrade di CMS.

Nel Capitolo \ref{cap:PrimoCapitolo} è quindi presentato il Large Hadron Collider e il suo principale esperimento multifunzionale CMS. In particolare è presentato nel dettaglio il funzionamento delle camere muoniche e dei principali rilevatori che hanno la funzione di rilevazione dei muoni. È discusso inoltre il sistema di Trigger di CMS e della sua importanza nel filtrare i dati generati dalle collisioni. Sono inoltre discussi i miglioramenti che CMS subirà con la \textit{Phase 2} che permetteranno una migliore ricostruzione dei muoni grazie ad un nuovo sistema di tracciamento direttamente nel Trigger Level 1 e infine è discussa la tecnica di Data Scouting introdotta con la \textit{Run 3} nella catena di Trigger.

Nel Capitolo \ref{cap:SecondoCapitolo} sono presentati e validati i risultati del sistema di Data Scouting introdotto con la Run 3, verificando la conformità dei dati del sistema: è presentato uno studio approfondito sul confronto tra candidati muoni emulati in software del sistema di tracking della regione di barrel, e i muoni estratti dal sistema di Data Scouting nello step successivo della catena di Trigger, esaminandone le differenze. Infine viene introdotta e studiata la \textit{matching efficiency} tra i muoni emulati e quelli della catena di Trigger, espandendo quindi la validazione del sistema di Data Scouting in funzione di alcune caratteristiche dei muoni.

Nel Capitolo \ref{cap:TerzoCapitolo} sono introdotte le Heavy Stable Charged Particles, HSCP, ed è presentato l'algoritmo utilizzato per la ricerca di eventi compatibili con i modelli alla base della teoria mostrando anche i principali risultati ottenuti.

Infine nel Capitolo \ref{cap:Conclusioni} verranno presentate le conclusioni ottenute dei risultati del presente studio. 

