\chapter{Introduzione}
\label{cap:Introduzione}

I muoni giocano un ruolo cruciale nei principali esperimenti effettuati nel Large Hadron Collider, nello specifico a CMS. Dal momento che sono le uniche particelle rilevabili che attraversano l'intero detector senza un'elevata perdita di energia per unità di lunghezza, i muoni sono largamente utilizzati per la rilevazione di fenomeni interessanti derivanti dalle collisioni di protoni in LHC. La rilevazione e la ricostruzione dei muoni è quindi un processo di fondamentale importanza a CMS.\newline
In questo studio sono presentate le proprietà dei candidati muoni ottenuti direttamente dal sistema di Trigger di primo livello a CMS, noto come L1T. Il sistema di Trigger a CMS riceve eventi ad un rate di 40MHz derivanti dalle collisioni di protoni e ha lo scopo di ridurre il volume di dati, mantenendo solamente quelli interessanti.
Come verrà discusso nel Capitolo \ref{cap:PrimoCapitolo} la selezione di eventi fisici da parte del sistema di Trigger introduce un bias, ovvero un pregiudizio sui dati fondato sulle leggi fisiche note, mascherando possibili informazioni che potrebbero portare alla conferma di teorie di fisica Oltre il Modello Standard. A questo scopo viene introdotto il sistema di Data Scouting nel L1T, ovvero un sistema che consente di raccogliere eventi derivanti dalle collisioni direttamente dalla catena di Trigger, seppur con una minore risoluzione, e lavorando parallelamente ad essa. Il sistema di Data Scouting nel L1T verrà implementato definitivamente a CMS con la \textit{Phase 2}, in contemporanea all'upgrade di LHC. In previsione della Phase 2, durante la Run 3 a CMS è stato introdotto un sistema di Data Scouting apposito a livello del L1T che consente la validazione e la sperimentazione di nuovi algoritmi da implementare con l'upgrade di CMS.

Nel Capitolo \ref{cap:PrimoCapitolo} verrà quindi introdotto il Large Hadron Collider e il suo principale esperimento multifunzionale CMS, locato a Cessy Francia nel punto di interazione 5 di LHC. Particolare attenzione verrà posta sulle camere muoniche, e sui principali rilevatori che permettono la rilevazione di muoni a CMS. Verrà discusso dettagliatamente anche del sistema di Trigger e del suo funzionamento nel filtrare i dati generati dalle collisioni. Verranno poi discussi i miglioramenti che CMS subirà con la \textit{Phase 2} che permetteranno una migliore ricostruzione dei muoni grazie ad un nuovo sistema di tracciamento direttamente nel Trigger Level 1. Infine sarà discussa la tecnica di Data Scouting introdotta con la \textit{Run 3} nella catena di Trigger.

Nel Capitolo \ref{cap:SecondoCapitolo} verranno presentati e validati i risultati del sistema di Data Scouting introdotto con la Run 3, verificando la conformità dei dati del Data Scouting. Verrà inoltre eseguito uno studio approfondito sul confronto tra candidati muoni emulati in software del sistema di tracking BMTF e i muoni del GMT, esaminandone le differenze. Verrà inoltre presentato un breve confronto con i dati raccolti dal sistema di Data Scouting nell'anno 2023 \cite{CERNsummerSchool}, evidenziandone le principali differenze. Infine sarà esplorata la \textit{matching efficiency} tra i muoni del BMTF e del GMT, espandendo quindi la validazione del sistema di Data Scouting.

Nel Capitolo \ref{cap:TerzoCapitolo} verranno introdotte le Heavy Stable Charged Particles, HSCP, e verrà invece introdotto l'algoritmo utilizzato per la ricerca di eventi compatibili con i modelli alla base della teoria mostrando anche i principali risultati ottenuti.

Infine nel Capitolo \ref{cap:Conclusioni} verranno presentate le conclusioni ottenute dai risultati del presente studio. 

