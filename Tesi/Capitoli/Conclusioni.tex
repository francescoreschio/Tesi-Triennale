\chapter{Conclusioni}
\label{cap:Conclusioni}

In questo studio sono state mostrate e le proprietà dei candidati muoni derivanti dal sistema di Trigger L1 dal sistema di Data Scouting introdotto con la Run 3 a CMS nella regione di barrel. È stato mostrato il filling scheme delle stubs nei 3564 bunch crossing del fascio di protoni ed è stata indagata la distribuzione della molteplicità di stubs per BX e per orbita. Inoltre è stata mostrata la distribuzione bidimensionale delle stubs nel corpo di CMS per stazioni, settori e wheel, ed è stato evidenziato come siano presenti gli stessi problemi presenti nell'anno 2023 nella raccolta di dati dei rilevatori nelle wheel più esterne nella prima stazione.

Dopo aver discusso come i segnali derivanti dai rilevatori DT e RPC nella regione di barrel vengono combinati dalle schede TwinMux, sono state discusse le principali differenze tra candidati muoni generati dal sistema BMTF utilizzando il Kalman filter e quelli emulati in software, verificando le principali proprietà di questi ultimi. Inoltre questi sono stati confrontati con i muoni estratti dalla catena di Trigger a livello del Global Muon Trigger dal sistema di Data Scouting, verificando che tra i muoni emulati e quelli estratti dalla catena di Trigger presentassero le stesse caratteristiche. Infine è stata studiata l'efficienza di match tra le due tipologie di muoni in funzione del momento trasverso, della qualità e della posizione nel piano della pseudorapidità, caratterizzando le proprietà dei muoni nelle varie regioni di matching efficiency. 


Nell'ultimo capitolo sono state introdotte le Heavy Stable Charged Particles ed è stata discussa la loro ricerca a CMS e di come la Phase 2 possa avere un ruolo importante nella ricerca di Nuova Fisica. In quest'ottica sono state evidenziate le principali firme sperimentali delle HSCP, quali un'anormale perdita di energia per unità di lunghezza e una velocità ridotta. Basandosi su quest'ultima particolarità è stato introdotto e applicato un algoritmo per la ricerca di particelle lente, ovvero particelle che percorrono le stazioni delle camere muoniche in un intervallo temporale elevato.\newline
Sono stati quindi caratterizzati i segnali delle stubs compatibili con i segnali di particelle lente, studiando la distribuzione spaziale delle stubs nei vari BX. Sono stati poi studiati i muoni generati dal sistema di Tracking della zona di barrel, caratterizzando prima le distribuzioni di momento e qualità dei singoli muoni formati da 3 e 4 stubs. Verificando che tra questi non vi sono particolari differenze, sono stati studiati invece i più interessanti eventi in cui sono presenti due muoni in BX adiacenti generati da stub compatibili con segnali di particelle lente: avendo confermato che la maggior parte degli eventi deriva da BX di collisione, è stata studiata la distribuzione delle differenze di momento e di posizione spaziale, verificando che queste particelle in BX adiacenti siano \textit{simili}, ovvero che possibilmente siano rappresentati dalla stessa particella.