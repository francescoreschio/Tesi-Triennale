\chapter{Conclusioni}
\label{cap:Conclusioni}

In questo studio sono state mostrate e le proprietà dei candidati muoni derivanti dal sistema di Trigger L1 del dal sistema di Data Scouting introdotto con la Run 3 a CMS. È stato mostrato il filling scheme delle stubs nei 3564 bunch crossing del fascio di protoni ed è stata indagata la distribuzione della molteplicità di stubs per BX e per orbita. Inoltre è stata mostrata la distribuzione bidimensionale delle stubs nel corpo di CMS per stazioni, settori e wheel, ed è stato evidenziato come siano presenti gli stessi problemi presenti nell'anno 2023 nella raccolta di dati dei rilevatori nelle wheel più esterne nella prima stazione. \newline
Dopo aver discusso delle proprietà dei candidati muoni emulati del kBMTF, questi sono stati confrontati con i muoni estratti dalla catena di Trigger a livello del Global Muon Trigger dal sistema di Data Scouting, verificandone la conformità, verificando inoltre la matching efficiency tra i muoni del kBMTF e del GMT in funzione del momento trasverso, della qualità e della posizione nel piano della pseudorapidità, caratterizzando le proprietà dei muoni nelle varie zone di matching efficiency.

Infine è stato applicato l'algoritmo per la ricerca di particelle lente, caratterizzando le proprietà spaziali delle stubs di questi eventi. Sono inoltre state esplorate le proprietà dei candidati muoni associati alle stubs compatibili con i segnali di particelle lente in un totale di ... minuti di presa dati a CMS.