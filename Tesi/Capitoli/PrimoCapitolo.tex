\chapter{Il progetto LHC:}


\section{LHC}
Formato da un anello di circonferenza pari a 27km, il Large Hadron Collider (LHC), situato al CERN in Ginevra, Svizzera, è il più grande acceleratore di particelle mai costruito, disegnato con lo scopo di studiare collisioni tra protoni con un energia nel centro di massa $\sqrt{s} = 13.6$ TeV e una luminosità istantanea nominale $L = 2 \times 10^{34} cm^{-2} s^{-1}$, corrispondente ad un rate di interazioni di 40MHz, ovvero una collisione ogni 25ns. L'intervallo temporale tra le collisioni è chiamato "bunch crossing, BX" ed è una unità di misura standardizzata quindi 1 BX = 25ns.

%Vengono fatti collidere pacchetti formati da $1.1 \times 10^{11}$ protoni, iniettati nell'anello dell'LHC con una energia di 450GeV grazie ad ulteriori acceleratori 

All'interno dell'LHC pacchetti (beam), formati da $1.1 \times 10^{11}$ protoni, circolano in due condotti differenti in direzioni opposte e collidono in quattro punti di interazione (IP) dove sono presenti i principali rivelatori dell'LHC: ATLAS (IP1), Alice (IP2), CMS (IP5) e LHCb (IP8). \\

I protoni subiscono una serie di fasi di accelerazione prima di essere immessi nell'LHC [LHC Machine]: una prima fase di accelerazione fino a 50MeV ad opera di un acceleratore lineare Linac, poi vengono accelerati fino a 1.4GeV dal Proton Synchrotron Booster (PSB), quindi a 28GeV dal Proton Synchrotron (PS) e infine dal Super Proton Synchrotron (SPS) a 450GeV. A questo punto i protoni sono iniettati nell'LHC dove verranno accelerati fino a 7TeV, collidendo quindi nei punti di intersezione con una energia nel centro di massa $\sqrt{s} \approx 14TeV$

L'LHC alterna periodi di fasi di attività e di raccolta dati (Run) con fasi di arresto in cui vengono effettuate opere di upgrade e di manutenzione [Sirunyan_2020_J._Inst._15_P10017.pdf]. Tra la Run 1, iniziata nel 2009 e finita nel 2013, e la Run 2, tra 2015 e 2018, l'energia del centro di massa per collisioni protone - protone è aumentata da 8 a 13 TeV e durante la Run 2 la luminosità istantanea ha subito un incremento, 




